\documentclass[12pt, a4paper, twoside]{article}

%%%%%%%%%%%%%%%%%%%%%%%%%%%%%
%% Packages
% System - Sprache
\usepackage[utf8x]{inputenc}
\usepackage[ngerman]{babel}
\usepackage[T1]{fontenc}
\usepackage{lmodern}
\usepackage{ucs}
\PrerenderUnicode{äüößÄÜÖ} 

\usepackage[colorlinks=true, pdfborder={0 0 0}, pdftex, breaklinks=true, linkcolor=blue, citecolor=orange , filecolor=purple , urlcolor=purple, linktocpage=true]{hyperref}

% Seiten Layout
\usepackage{fancyhdr}
\usepackage{geometry}
\usepackage{pdfpages}
\usepackage{minitoc}

% TIKZ
%\usepackage{tikz}
%\usepackage{00_Packages/tikzuml-v1.0b/tikz-uml}
%\usepackage{00_Packages/tikz-er2}
%\usepackage{00_Packages/pgf-pie-0.2.1/pgf-pie}
%\usepackage{uml}
%\usepackage{pgf}

% Inhalt
\usepackage{wrapfig} %floatfig
\usepackage{sidecap} %floatfig
\usepackage{float}
\usepackage{dpfloat}
\usepackage{algorithmicx}
\usepackage{algorithm}
\usepackage{algpseudocode}
\usepackage{prettyref}
\usepackage{titleref}
\usepackage{listings}
%\usepackage{glossaries}
\usepackage{nomencl}
\usepackage{cite}
\usepackage{graphicx}
\usepackage{verbatim}
\usepackage[font=small,labelfont=bf]{caption}
\usepackage[colorinlistoftodos]{todonotes}

% Mathe
%\usepackage{calc}
%\usepackage{ifthen}
%\usepackage{times}
\usepackage{amsmath}
%\usepackage{mathtools}

%%%%%%%%%%%%%%%%%%%%%%%%%%%%%
%% Farben
\definecolor{lightgrey}{gray}{.8}
\definecolor{lila}{rgb}{0.,0,0.5}

%%%%%%%%%%%%%%%%%%%%%%%%%%%%%
%% Seiten-Einstellungen
% Seitengrößen
\geometry{left=25mm, right=30mm, top=40mm, bottom=35mm}

% Zeilenabstand: 1,5
\renewcommand{\baselinestretch}{1.5}

% Einrückung bei Absatzbeginn
\parindent0ex

% Absatzgröße
\parskip3ex

%%%%%%%%%%%%%%%%%%%%%%%%%%%%%
%% Befehl ReWrites
% Zitate
\newcommand{\dz}[2]{"#1 {\renewcommand{\baselinestretch}{1}\footnote{#2}}"} % für direkte Zitate - \dz{Zitat}{Quelle}
\newcommand{\iz}[2]{#1 {\renewcommand{\baselinestretch}{1}\footnote{#2}}} % für indirekte Zitate - \iz{Zitat}{Quelle}


% Namensänderungen
\addto\captionsngerman{
\renewcommand{\lstlistingname}{Quellcode} % benennt "Listing" um
\renewcommand{\lstlistlistingname}{Quellcodeverzeichnis}
\renewcommand{\bibname}{Quellenverzeichnis} % benennt "Literatur" um
\renewcommand{\refname}{Quellenverzeichnis}}

\floatname{algorithm}{Algorithmus}

%% minitoc: einstellungen für minitocs
\mtcselectlanguage{german}
\mtcsettitle{secttoc}{Inhalt}
\mtcsetpagenumbers{secttoc}{off}
\mtcsetrules{secttoc}{off}


%%%%%%%%%%%%%%%%%%%%%%%%%%%%%
%% Referenzendefinitionen
% - full
% Abschnitte
\newrefformat{sec}{\textcolor{blue}{Kapitel} \titleref{#1} auf Seite \pageref{#1}}
\newrefformat{subsec}{\textcolor{blue}{Unterabschnitt} \titleref{#1} auf Seite \pageref{#1}}
\newrefformat{secmin}{\textcolor{blue}{Kapitel} \titleref{#1}}
\newrefformat{subsecmin}{\textcolor{blue}{Unterabschnitt} \titleref{#1}}
\newrefformat{anhang}{\textcolor{blue}{Anhang} ~\ref{#1} auf Seite \titleref{#1}}
\newrefformat{anhangmin}{\textcolor{blue}{Anhang} ~\ref{#1}}
% Abbildungen
\newrefformat{fig}{\textcolor{blue}{Abbildung}~\ref{#1} auf Seite \pageref{#1}}
\newrefformat{figmin}{\textcolor{blue}{Abbildung}~\ref{#1}}
% Tabellen
\newrefformat{tab}{\textcolor{blue}{Tabelle}~\ref{#1} auf Seite \pageref{#1}} 
\newrefformat{tabmin}{\textcolor{blue}{Tabelle}~\ref{#1}} 
% Quellcodeausschnitt
\newrefformat{lst}{\textcolor{blue}{Quellcodeausschnitt}~\ref{#1} auf Seite \pageref{#1}} 
\newrefformat{lstmin}{\textcolor{blue}{Quellcodeausschnitt}~\ref{#1}}
% Algorithmen
\newrefformat{alg}{\textcolor{blue}{Algorithmus}~\ref{#1} auf Seite \pageref{#1}} 
\newrefformat{algmin}{\textcolor{blue}{Algorithmus}~\ref{#1}}

%% Style Elemente
%\newcommand{\hsdesignbarfour}[1][0.99]{%
%	\begin{tikzpicture}[scale=#1]
%	\draw [hsblue2, fill=hsblue2] (0,0) -- (0.1667 \textwidth ,0.0136 \textwidth ) -- %(0.1667 \textwidth ,0.042 \textwidth) -- (0,0.042 \textwidth );