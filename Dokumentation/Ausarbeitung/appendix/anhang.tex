\section{Anhang}
\label{anhangmin:Anhang}

Dieser Teil des Dokuments bildet den Anhang. In diesem sind alle relevanten Informationen aufgeführt. 

\secttoc

\subsection{NAO Systemkonfigurationen}
\label{anhangmin:NAOSyskonfig}

\subsubsection{Anmeldedaten für die NAO Weboberfläche}

Um diverse Konfigurationen am Roboter NAO vornehmen zu können, werden folgende Anmeldeinformationen benötigt (Hinweise wie die Weboberfläche erreicht werden kann sind unter \cite{NAO-startup} zu finden):
\begin{itemize}
	\item Username: \texttt{nao}
	\item	Passwort: \texttt{Robert2014}
\end{itemize} 

\subsection{Inhalte der CD}

Auf der CD befinden sich alle Daten, die während der Entwicklung des Text-Datenbank Management Systems und dem Anfertigen der Ausarbeitung entstanden sind. Dabei können die Daten in drei verschiedene Kategorien eingeteilt werden:
\begin{enumerate}
	\item TextDB-Projektentwicklung
	\item Ausarbeitung 
	\item Quellennachweise
\end{enumerate}

Die Strukturierung der CD ist im nachstehenden Verzeichnisbaum dargestellt:

%%%%%%%%%%%%%%%%%%%%%%%%%%%%%%%%%%%%%%%
%% Tikz CD Verzeichnisstruktur
\begin{figure}[h!]
	\centering
		%\input{00_Anhang/Bilder/tikz/cd_verzeichnis.tex}
	%\caption[CD-Verzeichnisstruktur]{CD-Verzeichnisstruktur}
	%\label{figmin:CD_Verzeichnis}
\end{figure}
%%%%%%%%%%%%%%%%%%%%%%%%%%%%%%%%%%%%%%%

