%%%%%%%%%%%%%%%%%%%%%%%%%%%%%%%%%%%%%%%
%%%%%%%%%%%%%%%%%%%%%%%%%%%%%%%%%%%%%%%	
\section{Anhang}
	\label{anhangmin:Anhang}

% Mini TOC generieren
\secttoc

%%%%%%%%%%%%%%%%%%%%%%%%%%%%%%%%%%%%%%%
%%%%%%%%%%%%%%%%%%%%%%%%%%%%%%%%%%%%%%%	
\subsection{NAO Systemkonfigurationen}
	\label{anhangmin:NAOSyskonfig}

%%%%%%%%%%%%%%%%%%%%%%%%%%%%%%%%%%%%%%%
%%%%%%%%%%%%%%%%%%%%%%%%%%%%%%%%%%%%%%%	
\subsubsection{Anmeldedaten für die NAO Weboberfläche}

Um diverse Konfigurationen am Roboter NAO vornehmen zu können, werden folgende Anmeldeinformationen benötigt (Hinweise wie die Weboberfläche erreicht werden kann sind unter 


% TODO -> NAO startup
%\cite{NetS} zu finden):


\begin{itemize}
	\item Username: \texttt{nao}
	\item	Passwort: \texttt{Robert2014}
\end{itemize} 


%%%%%%%%%%%%%%%%%%%%%%%%%%%%%%%%%%%%%%%
%%%%%%%%%%%%%%%%%%%%%%%%%%%%%%%%%%%%%%%	
\subsection{Einrichten des Python 2.7 naoqi SDKs}

Die Installation und Konfiguration efolgte nach den Empfehlungen von Aldebaran, welche unter (QUELLE) zu finden ist.
Da dieses Projekt lediglich mit Hilfe eines Apple MacBook Pro durchgeführt wurde, wurde selbstverständlich darauf geachtet nur die für Mac OS X geltenden Vorgaben umzusetzen.

Zunächst empfiehlt Aldebaran, um späteres Fehlverhalten der Software auszuschließen, das original \textsc{Python 2.7 SDK} zu verwenden, welches sich bereits zum Auslieferungszustand auf dem MacBook befindet.
Die \textsc{Python 2.7 Shell} kann unter Mac OS X unter folgendem Pfad gefunden werden:

\texttt{/usr/bin/python}

Um Python-Skripte für den NAO entwiklen zu können benötigt man außerdem das \textsc{PyNAOqi Python 2.7 SDK} von Aldebaran,
welches ausschließlich erworben werden kann wenn ein verifizierter Aldebaran Account vorhanden ist. Darüber hinaus muss ein NAO-Roboter mit diesem Account verbunden worden sein.

Path setzen

et...

Test


%%%%%%%%%%%%%%%%%%%%%%%%%%%%%%%%%%%%%%%
%%%%%%%%%%%%%%%%%%%%%%%%%%%%%%%%%%%%%%%	
\subsection{Inhalte der CD}

Auf der CD befinden sich alle Daten, die während der Entwicklung des Text-Datenbank Management Systems und dem Anfertigen der Ausarbeitung entstanden sind. Dabei können die Daten in drei verschiedene Kategorien eingeteilt werden:
\begin{enumerate}
	\item TextDB-Projektentwicklung
	\item Ausarbeitung 
	\item Quellennachweise
\end{enumerate}

Die Strukturierung der CD ist im nachstehenden Verzeichnisbaum dargestellt:

%%%%%%%%%%%%%%%%%%%%%%%%%%%%%%%%%%%%%%%
%% Tikz CD Verzeichnisstruktur
\begin{figure}[h!]
	\centering
		%\input{00_Anhang/Bilder/tikz/cd_verzeichnis.tex}
	%\caption[CD-Verzeichnisstruktur]{CD-Verzeichnisstruktur}
	%\label{figmin:CD_Verzeichnis}
\end{figure}
%%%%%%%%%%%%%%%%%%%%%%%%%%%%%%%%%%%%%%%

